%!TEX TS-program = xelatex
%!TEX encoding = UTF-8 Unicode
% Awesome CV LaTeX Template for CV/Resume
%
% This template has been downloaded from:
% https://github.com/posquit0/Awesome-CV
%
% Author:
% Claud D. Park <posquit0.bj@gmail.com>
% http://www.posquit0.com
%
%
% Adapted to be an Rmarkdown template by Mitchell O'Hara-Wild
% 23 November 2018
%
% Template license:
% CC BY-SA 4.0 (https://creativecommons.org/licenses/by-sa/4.0/)
%
%-------------------------------------------------------------------------------
% CONFIGURATIONS
%-------------------------------------------------------------------------------
% A4 paper size by default, use 'letterpaper' for US letter
\documentclass[11pt,a4paper,]{awesome-cv}

% Configure page margins with geometry
\usepackage{geometry}
\geometry{left=1.4cm, top=.8cm, right=1.4cm, bottom=1.8cm, footskip=.5cm}


% Specify the location of the included fonts
\fontdir[fonts/]

% Color for highlights
% Awesome Colors: awesome-emerald, awesome-skyblue, awesome-red, awesome-pink, awesome-orange
%                 awesome-nephritis, awesome-concrete, awesome-darknight

\definecolor{awesome}{HTML}{e62e00}

% Colors for text
% Uncomment if you would like to specify your own color
% \definecolor{darktext}{HTML}{414141}
% \definecolor{text}{HTML}{333333}
% \definecolor{graytext}{HTML}{5D5D5D}
% \definecolor{lighttext}{HTML}{999999}

% Set false if you don't want to highlight section with awesome color
\setbool{acvSectionColorHighlight}{true}

% If you would like to change the social information separator from a pipe (|) to something else
\renewcommand{\acvHeaderSocialSep}{\quad\textbar\quad}

\def\endfirstpage{\newpage}

%-------------------------------------------------------------------------------
%	PERSONAL INFORMATION
%	Comment any of the lines below if they are not required
%-------------------------------------------------------------------------------
% Available options: circle|rectangle,edge/noedge,left/right

\name{Aaron Wenger}{}

\position{Graduate Student}
\address{Kalamazoo, Michigan}

\email{\href{mailto:aaron.wenger@protonmail.com}{\nolinkurl{aaron.wenger@protonmail.com}}}
\github{ace-wenger}

% \gitlab{gitlab-id}
% \stackoverflow{SO-id}{SO-name}
% \skype{skype-id}
% \reddit{reddit-id}

\quote{I am a Ph.D.~candidate in the Mallinson Institute of Science
Education program. I have a passion for 1) teaching and learning at all
levels and subjects, particularly biology in higher education and 2) the
role and nature of evidence for decisions in education. Example work can
be found on my Github page in the `Portfolio' repository.}

\usepackage{booktabs}

\providecommand{\tightlist}{%
	\setlength{\itemsep}{0pt}\setlength{\parskip}{0pt}}

%------------------------------------------------------------------------------



% Pandoc CSL macros
% definitions for citeproc citations
\NewDocumentCommand\citeproctext{}{}
\NewDocumentCommand\citeproc{mm}{%
  \begingroup\def\citeproctext{#2}\cite{#1}\endgroup}
\makeatletter
 % allow citations to break across lines
 \let\@cite@ofmt\@firstofone
 % avoid brackets around text for \cite:
 \def\@biblabel#1{}
 \def\@cite#1#2{{#1\if@tempswa , #2\fi}}
\makeatother
\newlength{\cslhangindent}
\setlength{\cslhangindent}{1.5em}
\newlength{\csllabelwidth}
\setlength{\csllabelwidth}{3em}
\newenvironment{CSLReferences}[2] % #1 hanging-indent, #2 entry-spacing
 {\begin{list}{}{%
  \setlength{\itemindent}{0pt}
  \setlength{\leftmargin}{0pt}
  \setlength{\parsep}{0pt}
  % turn on hanging indent if param 1 is 1
  \ifodd #1
   \setlength{\leftmargin}{\cslhangindent}
   \setlength{\itemindent}{-1\cslhangindent}
  \fi
  % set entry spacing
  \setlength{\itemsep}{#2\baselineskip}}}
 {\end{list}}
\usepackage{calc}
\newcommand{\CSLBlock}[1]{\hfill\break\parbox[t]{\linewidth}{\strut\ignorespaces#1\strut}}
\newcommand{\CSLLeftMargin}[1]{\parbox[t]{\csllabelwidth}{\strut#1\strut}}
\newcommand{\CSLRightInline}[1]{\parbox[t]{\linewidth - \csllabelwidth}{\strut#1\strut}}
\newcommand{\CSLIndent}[1]{\hspace{\cslhangindent}#1}

\begin{document}

% Print the header with above personal informations
% Give optional argument to change alignment(C: center, L: left, R: right)
\makecvheader

% Print the footer with 3 arguments(<left>, <center>, <right>)
% Leave any of these blank if they are not needed
% 2019-02-14 Chris Umphlett - add flexibility to the document name in footer, rather than have it be static Curriculum Vitae
\makecvfooter
  {April, 2024}
    {Aaron Wenger~~~·~~~CV}
  {\thepage}


%-------------------------------------------------------------------------------
%	CV/RESUME CONTENT
%	Each section is imported separately, open each file in turn to modify content
%------------------------------------------------------------------------------



\section{Education}\label{education}

\begin{cventries}
    \cventry{Ph.D. Student in Science Education}{Western Michigan University}{Kalamazoo, Michigan}{2017-present}{\begin{cvitems}
\item Ph.D. candidate, currently completing dissertation project with a defense planned for summer 2024
\item Advised and chaired by Dr. William Cobern
\end{cvitems}}
    \cventry{Projects/Manuscripts in Progress (Working Titles)}{}{}{}{\begin{cvitems}
\item Dissertation: Beyond Average Effects in Education Research: Explaining heterogeneity of concept mapping research in science education through meta-regression modeling \begin{enumerate} \item[] \textbf{Article 1:} Challenges and Solutions for Knowledge Accumulation in Science Education Research \item[] \textbf{Article 2:} Concept Mapping in Biology Education: \textit{A Systematic Review and Meta-Analysis} \item[] \textbf{Article 3:} Explaining Heterogeneity in Science Education Research: \textit{Comparing machine learning models with a priori meta-regression models} \end{enumerate}
\item Mapping the Homeschooling Literature: \textit{A Scoping Review and Source Analysis}
\end{cvitems}}
    \cventry{M.A. in Biological Sciences}{Western Michigan University}{Kalamazoo, Michigan}{2017-2021}{\begin{cvitems}
\item Master's Thesis Project: Engineered Flagellin Disulfide Variants in Salmonella typhimurium
\item Advised by Dr. Brian Tripp
\end{cvitems}}
    \cventry{B.S. in Biology-Health Sciences - minors in Chemistry, Ancient (History) Studies}{Cornerstone University}{Grand Rapids, Michigan}{2011-2015}{\begin{cvitems}
\item Recipient of the Ronald Meyer Academic Scholar, Dean's list (6 of 7 semesters), and President's Scholarship
\item Senior Thesis Project: a meta-study of the neural crest as a mechanism for vertebrate phenotypic diversity
\item Internship: Hesse Memorial Archaelogical Laboratory, learned and applied zooarcheological techniques with animal bone remains
\end{cvitems}}
\end{cventries}

\section{Teaching Experience}\label{teaching-experience}

\begin{cventries}
    \cventry{Teaching Assistant}{Western Michigan University}{Kalamazoo, Michigan}{2017-2021}{\begin{cvitems}
\item Instructor of record for lecture/lab science content courses for elementary education majors
\item CHEM 2800: Physical Science for Elementary Educators: a inquiry-based, activity-centered course covering basic chemical and physical science principles
\item BIOS 1700 in-person: Life Science for Non-Majors: a laboratory-lecture-based content course covering major topics in the life sciences
\item BIOS 1700 virtual: An adaptation of BIOS 1700 to an online synchronous format, for which I was responsible to design and implement
\item GEOG 1900: Exploring Earth Science, the Atmosphere: A laboratory-based course covering basic earth science principles with an emphasis on the atmosphere; taught as a virtual, partially synchronous course
\end{cvitems}}
    \cventry{Lecturer for Pre-Med Intiative}{}{}{2018}{\begin{cvitems}
\item A student-led program for MCAT exam preparation
\end{cvitems}}
    \cventry{Science Teacher}{Friday Addition (FA) and Homeschool Ancillary Program (HsAP)}{Michigan}{2015-2017}{\begin{cvitems}
\item Developed and taught 9th grade biology, 7th grade general science, and 10th grade chemistry.
\end{cvitems}}
    \cventry{College Tutor and Laboratory Assistant}{Cornerstone University}{Grand Rapids, Michigan}{2012-2015}{\begin{cvitems}
\item Tutored undergraduate genetics, chemistry, physics, math, and history
\item Assisted in stockroom, ordering materials, organizing activities, and grading laboratory reports
\end{cvitems}}
    \cventry{Areas of Expertise for Teaching and Public Engagement}{}{}{}{\begin{cvitems}
\item 1) Scientific methods, experimental design, causal inference, and philosophy of science
\item 2) Introductory and advanced biology subjects, particularly microbiology and molecular biology
\item 3) History of educational psychology and science education
\item 4) Statistical and computational research methods using R, especially computational reproducibility
\end{cvitems}}
\end{cventries}

\section{Research Experience}\label{research-experience}

\begin{cventries}
    \cventry{Research Interests}{}{}{}{\begin{cvitems}
\item 1) Meta-research methods, especially meta-analysis and bibliometrics
\item 2) Computational reproducibility and open science practices
\item 3) Role of evidence in educational policy and practice
\end{cvitems}}
    \cventry{Graduate Research Assistant in Science And Mathematics Program Improvement (SAMPI)}{Western Michigan University}{Kalamazoo, MI}{2021-2023}{\begin{cvitems}
\item Assisted in program evaluation for clients including NSF-funded Professional development for Emerging Education Researchers (PEER) field school,  Kalamazoo Scholars Program, and the Michigan STEM Network (MiSTEM
\item Created protocols and evaluation tools (e.g., Qualtric surveys and interview questions)
\item Conducted quantitative and qualitative analysis of numerical, ordinal, and textual data
\item Wrote internal and external reports summarizing findings
\end{cvitems}}
\end{cventries}

\section{Portfolio and Further
Education}\label{portfolio-and-further-education}

\subsection{Software for Statistics and Data
Science}\label{software-for-statistics-and-data-science}

\begin{cvhonors}
    \cvhonor{}{\textbf{R}: substantial programming experience with base R, the Rstudio IDE, and common packages such as 'ggplot' (See \href{https://github.com/ace-wenger/ConceptMapping_inBioEd}{\textit{ConceptMapping-inBioEd}})}{}{}
    \cvhonor{}{\textbf{Git and GitHub}: substantial experience creating and mangaging projects using Git version control and the GitHub collaboration platform (See my GitHub account for several public projects)}{}{}
    \cvhonor{}{\textbf{Analysis Pipeline Tools}: substantial experience implementing data analysis pipelines with the `targets` and `renv` R packages (See \href{https://github.com/ace-wenger/ConceptMapping_inBioEd}{\textit{ConceptMapping-inBioEd}})}{}{}
    \cvhonor{}{\textbf{SPSS and SAS}: minor programming experience with both (See \href{https://github.com/ace-wenger/Portfolio/sas}{\textit{Portfolio/sas}})}{}{}
    \cvhonor{}{\textbf{Excel VBA}: minor programming experience (See \href{https://github.com/ace-wenger/Portfolio/vba}{\textit{Portfolio/vba}})}{}{}
\end{cvhonors}

\subsection{Software for Documentation and
Reporting}\label{software-for-documentation-and-reporting}

\begin{cvhonors}
    \cvhonor{}{\textbf{Microsoft Office Suite}: extensive experience with Word, PowerPoint, Excel, Outlook, and Teams }{}{}
    \cvhonor{}{\textbf{Quarto}: substantial experience creating reports and presentations }{}{}
    \cvhonor{}{\textbf{Rmarkdown}: substantial experience creating data analysis notebooks, reports, and other documents such as this CV }{}{}
    \cvhonor{}{\textbf{LaTeX}: minor working experience using the TeXworks IDE such as in this CV }{}{}
\end{cvhonors}

\subsection{Other Software}\label{other-software}

\begin{cvhonors}
    \cvhonor{}{\textbf{Qualtrics}: substantial experience in creating survey forms, distributing to program participants, and processing results }{}{}
    \cvhonor{}{\textbf{Google Forms}: substantial experience in creating survey forms and processing results }{}{}
    \cvhonor{}{\textbf{Abstrackr and MetaReviewer}: substantial experience in these platforms for meta-research studies }{}{}
\end{cvhonors}

\subsection{Workshops and Online
Courses}\label{workshops-and-online-courses}

\begin{cvhonors}
    \cvhonor{}{\textbf{Instats: Confirmatory factor analysis and structural equation modeling in R}: Taught by Michael Zyphur and with certificate of completion}{}{2023}
    \cvhonor{}{\textbf{Instats: Meta-analytic structural equation modeling}: Taught by Mike Cheung and with certificate of completion}{}{2023}
    \cvhonor{}{\textbf{Evidence Synthesis and Meta-Analysis in R (ESMAR) Conference}: several workshops including: Advanced GitHub, Screening studies for eligibility in evidence syntheses}{}{2023}
    \cvhonor{}{\textbf{Excel VBA Programming}: video course introducing VBA programming and various applications such as macro implementation, userforms, and webscraping}{}{2022}
    \cvhonor{}{\textbf{Research Transparency Online Course}: put on by the Berkeley Initiative for Transparency in the Social Sciences (BITSS)}{}{2022}
    \cvhonor{}{\textbf{Reproducible Research Tutorial Series}: online course by Dr. Schloss of the University of Michigan, supported by NIH}{}{2022}
    \cvhonor{}{\textbf{Instats: Path analysis with interactions and indirect effects in R}: Taught by Michael Zyphur}{}{2022}
    \cvhonor{}{\textbf{Bibliometrics Training Series}: put on by the NIH Library}{}{2021}
    \cvhonor{}{\textbf{AERA-ICPSR PEERS}: attended several in this workshop series including: Modern Meta-analysis, Cutting-edge Quatitative and Computational Methods for STEM Education, and Introduction to qualitative meta-synthesis methods}{}{2020-21}
    \cvhonor{}{\textbf{Introduction to Systematic Reivew and Meta-Analysis}: a John Hopkins University course hosted by Coursera}{}{2020}
\end{cvhonors}

\section{Grants and Professional
Experience/Service}\label{grants-and-professional-experienceservice}

\begin{cvhonors}
    \cvhonor{}{\textbf{Graduate Student Panel Reviewer} \newline Reviewed five papers each for Division D (Measurement \& Research Methodologies) and SIG-SRMA (Systematic Review and Meta-Analysis special interest group)}{AERA Annual Meeting}{2024}
    \cvhonor{}{\textbf{Graduate Student Research Grant} \newline Secured for Science Education Research Project}{Western Michigan University}{2021}
    \cvhonor{}{\textbf{Graduate Student Research Grant} \newline Institutional grant secured for biology master thesis project}{Western Michigan University}{2019}
\end{cvhonors}

\section{Presentations and
Publications}\label{presentations-and-publications}

\phantomsection\label{refs-9d4a94543a88655e382d9ea4bed7b91a}
\begin{CSLReferences}{0}{0}
\bibitem[\citeproctext]{ref-WCrcm2023}
\CSLLeftMargin{1. }%
\CSLRightInline{Wenger, A., \& Cobern, W. (2023). \emph{Replication of
concept mapping research in biology education: A systematic review and
meta-analysis.} {[}Conference{]}. Michigan Academy of Science Arts and
Letters Annual Conference, Berrien Springs, Michigan.
\url{https://ace-wenger.quarto.pub/masal23-concept-mapping/}}

\bibitem[\citeproctext]{ref-WWeef2023}
\CSLLeftMargin{2. }%
\CSLRightInline{Williams, C., \& Wenger, A. (2023). \emph{Evaluating the
effects of field schools on emerging STEM education researchers.}
{[}Conference{]}. Michigan Academy of Science Arts and Letters Annual
Conference, Berrien Springs, Michigan.}

\end{CSLReferences}



\end{document}
